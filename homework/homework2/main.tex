\documentclass{exam}
\usepackage{../../shah311}

\hypersetup{colorlinks=true, linktoc=section, linkcolor=blue}

\pagestyle{headandfoot}
\firstpageheadrule
\runningheadrule
\firstpageheader{Prof. Rong \\ Real Analysis}{Homework 1}{Jeevan Shah}
\runningheader{Real Analysis \\ Homework 1}{}{Shah}
\firstpagefooter{}{}{}
\runningfooter{ }{\thepage}{ }

\printanswers

\begin{document}
\colorbox{red}{\underline{$1.4.4$}:}
Let $a < b$ be real numbers and consider the set $T = \QQ \cap [a, b]$. Show $\sup T = b$
\begin{solution}
    \begin{proof}
        By definition of $T$, $b$ must be an upper bound for $T$. Now, for any other upper bound of $T$, call it $c$, we must have $c \geq b$. This is because if $c < b$ then $c$ cannot be an upper bound for if $b \in T$ (i.e $b \in \QQ$) then $c$ is not greater than or equal to every element of $T$, and if $b \not\in T$ (i.e $b$ is irrational), then by the density of $\QQ$ in $\RR$, there exists a number $r \in \QQ$ such that $c < r < b$ so $r \in T$, which again means that $c$ cannot be an upper bound. Thus $b$ is least upper bound so $\sup T = b$.
    \end{proof}    
\end{solution}

\colorbox{red}{\underline{$1.4.5$}:}
Prove that for any two real numbers $a < b$, there exists an irration number, $t$, such that $a < t < b$ by considering the real numbers $a - \sqrt{2}$ and $b - \sqrt{2}$. 
\begin{solution}
    \begin{proof}
        By the density of $\QQ$ in $\RR$ there exists $r \in \QQ$ such that 
        \[
            a - \sqrt{2} < r < b - \sqrt{2}.
        \]
        So, 
        \[
            a < r + \sqrt{2} < b,
        \]
        but $r + \sqrt{2}$ is irrational (since a rational number plus an irrational number will always be irrational - by assumption of exercise 1.4.1), completing the proof.
    \end{proof}
\end{solution}

\colorbox{red}{\underline{$1.4.8$}:}
Give an example of each or state that the request is impossible. When a request is impossible proivde a compelling arugment for why this is the case.
\begin{parts}
    \part Two sets $A$ and $B$ with $A \cap B = \varnothing, \sup A = \sup B, \sup A \not\in A$ and $\sup B \not\in B$.
    \begin{solution}
        Consider $A = \braces{x \in \QQ \mid 0 < x^2 < 2}$ and $B = \braces{x \in \RR \setminus \QQ \mid 0 < x^2 < 2}$. Then $A \cap B = \varnothing$, $\sup A = \sup B = \sqrt{2}$ and $\sup A \not\in A$ and $\sup B \not\in B$.
    \end{solution}

    \part A sequence of nested open interval $J_1 \supseteq J_2 \supseteq J_3 \supseteq \cdots$ with $\bigcap_{n=1}^{\infty}J_n$ nonempty but containing only a finite number of elements.
    \begin{solution}
        Let $J_n = (1 - \frac{1}{n}, 1 + \frac{1}{n})$ such that each $J_n$ is nested and open. Then 
        \[
            \bigcup_{n=1}^{\infty} J_n = \braces{1} 
        \]
        which is nonempty and only contains a single element.
    \end{solution}

    \part A sequence of nested unbounded closed interval $L_1 \supseteq L_2 \supseteq L_3 \supseteq \cdots$ with $\bigcap_{n=1}^{\infty} L_n = \varnothing$.
    \begin{solution}
        Consider $L_n = [n, \infty)$. Then
        \[
            \bigcap_{n=1}^{\infty} L_n = \varnothing 
        \]
        since eventually $n > x$ for all $x \in \RR$.
    \end{solution}

    \part A sequence of closed bounded (not necesssarily nested) interval $I_1, I_2, I_3, \ldots$ with the property that $\bigcap_{n=1}^{N}I_n \neq \varnothing$ for all $N \in \NN$, but $\bigcap{n=1}^{\infty} I_n = \varnothing$.
    \begin{solution}
        This is impossible. We define $K_n = I_1 \cap I_2 \cap \cdots \cap I_n$. Then each $K_n$ is closed and bounded since each $I_n$ is also closed and bounded. We cannow apply the nested interval theorem since $K_1 \supseteq K_2 \supseteq K_3 \supseteq \cdots$, 
        \[
            \bigcap_{n=1}^{\infty} K_n = \bigcap_{n=1}^{\infty} I_n \neq \varnothing.
        \]
    \end{solution}
\end{parts}

\colorbox{red}{\underline{$1.5.9$}:}
\begin{parts}
    \part Show that $\sqrt{2}, \sqrt[3]{2},$ and $\sqrt{2} + \sqrt{3}$ are algebraic
    \begin{solution}
        For the polynomial $x^2 - 2 = 0$, we know that $\sqrt{2}$ is a root, so $\sqrt{2} \in \mathbb{A}$. For the polynomial $x^{3} - 2 = 0$, we know that $\sqrt[3]{2}$ is a root, so $\sqrt{2} \in \mathbb{A}$. For the polynomial $x^4 - 10x^2 + 1 = 0$, we know that $\sqrt{2} + \sqrt{3}$ is a root, so $\sqrt{2} + \sqrt{3} \in \mathbb{A}$. Note that we found this polynomial by considering $x - (\sqrt{2} + \sqrt{3}) = 0$ and then solving for $x$ before squaring the entire polynomial twice over to find integer coefficients.
    \end{solution}

    \part Fix $n \in \NN$, and let $A_n$ be the algebraic numbers obtained as roots of polynomials with integer coefficients that have degree $n$. Using the fact that every polynomial has a finite number of roots, show that $A_n$ is countable.
    \begin{solution}
        \begin{proof}
            We start by representing each polynomial with integer coefficients as a tuple of coefficients: 
            \[
                (a_n, a_{n-1}, \ldots, a_1, a_0)
            \]
            and consider the set of all polynomials with integer coefficients in this manner. However, this is the same as the set $\ZZ^{n+1}$ and since the product of a finite number of countable sets is also countable, the set of all polynomials with integer coefficients must also be countable. Since this set is countable and each polynomial of degree $n$ can have at most $n$ roots, it follows that $A_n$ must also be countable (by being the union of a countable amount of sets with finite elemnts).
        \end{proof}
    \end{solution}

    \part Now, argue that the set of all algebraic numbers is countable. What may we conclude about the set of transcendental numbers?
    \begin{solution}
        Since the set of algebraic numbers, $\mathbb{A}$ is countable, the set of transcendentals, $\RR \setminus \mathbb{A}$, must be uncountable. This is because every real number is either algebraic or transcendental, so $A \cup (\RR \setminus \mathbb{A}) = \RR$. Since $\RR$ is uncountable, $\RR \setminus \mathbb{A}$ must be uncountable as well since the union of two countable sets cannot be uncountable.
    \end{solution}
\end{parts}

\colorbox{red}{\underline{$1.5.11$}:}
\begin{parts}
    \part Explain how partitioning $X$ and $Y$ in such a way that $f$ maps $A$ onto $B$ and $g$ maps $B'$ onto $A'$ would lead to a proof that $X \sim Y$.
    \begin{solution}
        Since $g$ is both onto and $1-1$ we know that $g^{-1}: A' \to B'$ exists. We define $h: X \to Y$ as follows: 
        \[
            h(x) = \begin{cases}
                f(x) \quad& \text{if } x \in A \\
                g^{-1}(x) \quad& \text{if } x \in A' 
            \end{cases}.
        \]
        Since $f$ and $g^{-1}$ are bijections, $h$ must be one as well. Thus $X \sim Y$.
    \end{solution}

    \part Set $A_1 = X \setminus g(Y)$ and inductively define a sequence of sets by letting $A_{n+1} = g(f(A_n))$. Show that $\braces{A_n \mid n \in \NN}$ is a pairwise disjoint collection of subsets of $X$, while $\braces{f(A_n) \mid n \in \NN}$ is a similar collection in $Y$.
    \begin{solution}
        If $A_1 = \varnothing$ then $X = g(Y)$ so $g$ is onto and thus a bijection which implies $X \sim Y$. In the case $A_1 \neq \varnothing$ we may move on to induction. For the base case $A_1 \cap g(Y) = \varnothing$ by definition of $A_1$, so $A_n \cap A_1 = \varnothing$ for all $n > 1$ since $A_n = g(f(A_{n-1})) \subseteq g(Y)$. We now suppose that $A_m \cap A_n = \varnothing$ for some $n < m$. Then $f(A_n) \cap f(A_m) = \varnothing$ since $f$ is $1-1$. So, $A_{n+1} = g(f(A_n)) \cap g(f(A_m)) = A_{m+1}$ is empty and thus $\braces{A_n \mid n \in \NN}$ is pairwise disjoint. It follows from this that $\braces{f(A_n) \mid n \in \NN}$ is also pairwise disjoint in $Y$.
    \end{solution}

    \part Let $A = \bigcup_{n=1}^{\infty}A_n$ and $B = \bigcup_{n=1}^{\infty} f(A_n)$. Show that $f$ maps $A$ onto $B$.
    \begin{solution}
        Let $y \in B$. By definition of union, $y \in f(A_n)$ for some $n$. There exists $x \in A_n$ such that $f(x) = y$ by definition of $f(A_n)$. Since $A_n \subseteq A$, we know that $x \in A$. Thus, we have found $x \in A$ such that $f(x) = y$ for some $y \in B$ which implies that $f$ is onto from $A$ to $B$.
    \end{solution}

    \part Let $A' = X \setminus A$ and $B' = Y \setminus B$. Show $g$ maps $B'$ onto $A'$.
    \begin{solution}
        We will show that $g(B') = A'$. Suppose that $x \in g(B')$. Then $x = g(y)$ for some $y \in B'$. If $x \in A$ then $x \in A_n$ for some $n$. If $n = 1$ then $x \not\in g(Y)$ which contradicts the assumption that $x \in g(B') \subseteq g(Y)$. If $n > 1$ then $x \in A_n = g(f(A_{n-1}))$. Because $g$ is 1-1, $y = f(A_{n-1}) \subseteq B$ which contradicts the fact that $y \not\in B$. So, $x \in A'$. \\ 
        Now, let $x \in A'$. Then $x \not\in A$ so $x \not\in A_1 = X \setminus g(Y)$. So $x \in g(Y)$ which implies that $x = g(y)$ for some $y \in Y$. If $y \in B$, then $y \in f(A_n)$ for some $n$. Then $x = g(y) \in g(f(A_n)) = A_{n+1}$. But then $x \in A_{n+1} \subseteq A$ which implies $x \in A$ which is a contradiction. Thus $y \in B'$. \\
        So, since $g(B') = A'$, it must be that $g$ maps $B'$ onto $A'$. 
    \end{solution}
\end{parts}

\colorbox{red}{\underline{$1.6.1$}:}
Show that $(0, 1)$ is uncountable if, and only if, $\RR$ is uncountable.
\begin{solution}
    \begin{proof}
        Forwards: Assume $\RR$ is uncountable and consider the function 
        \[
            f(x) = \frac{e^{x}}{e^{x} + 1}.
        \]
        Then, $f$ is a bijective function with a domain of $\RR$ and a range of $(0, 1)$ since $f(x) \neq 0$ and $f(x) \neq 1$ for all $x \in \RR$. Thus, since there exists a bijection from $\RR$ to $(0, 1)$, we have that $\RR \sim (0, 1)$. \\ 
        Backwards: If $(0, 1)$ is uncountable it immediately follows that $\RR$ must be uncountable since if $\RR$ was countable, then every (infite) subset must be countable. However, this contradicts our assumption and so $\RR$ must be countable. 
    \end{proof}
\end{solution}

\newpage 

\colorbox{red}{\underline{$1.6.9$}:}
Show that $\mathcal{P}(\NN) \sim \RR$.
\begin{solution}
    \begin{proof}
        We will show that there exist injective functions $f: \mathcal{P}(\NN) \to \RR$ and $g: \RR \to \mathcal{P}(\NN)$. We define $f$ as follows: for every subset $S \subseteq \NN$, take $f(S)$ as the base-$10$ decimal representation $0.b_{1}b_{2}b_{3}\ldots$ where $b_n = 1$ if, and only if, $n \in S$ and $b_n = 0$ otherwise. Unique subsets will have unique images under $f$ so this function is injective. We define $g: \RR \to \mathcal{P}(\NN)$ as follows: for every number $x \in (0, 1)$ map the binary representation of $x=c_{1}c_{2}c_{3}\ldots$, where $c_n \in \braces{0, 1}$, to a subset $S \in \mathcal{P}(\NN)$ where $n \in S$ if, and only if, $c_n = 1$. This results in $g$ being injective since each binary representation is unique. It follows that $\mathcal{P}(\NN) \sim \RR$.
    \end{proof}    
\end{solution}

\colorbox{red}{\underline{$1.6.10$}:}
\begin{parts}
    \part Is the set of all functions from $\braces{0, 1}$ to $\NN$ countable or uncountable?
    \begin{solution}
        The set of all functions from $\braces{0, 1}$ to $\NN$ is countable. This is because for any $f: \braces{0, 1} \to \NN$, we can define a map to $\NN \times \NN$ given by $(f(0), f(1))$. Since each ordered pair $(a, b) \in \NN$ will appear only once (when $f(0) = a$ and $f(1) = b$), this is a bijection and thus the set of all functions is countable.
    \end{solution}

    \part Is the set of all function from $\NN$ to $\braces{0, 1}$ countable or uncountable?
    \begin{solution}
        The set of all functions from $\NN$ to $\braces{0, 1}$ is uncountable. This is because any function $f: \NN \to \braces{0, 1}$ must assign each $n \in \NN$ to either $0$ or $1$, so for any such function we can associate it with a subset $A \subseteq \NN$ such that $n \in A$ if, and only if, $f(n) = 1$. This is a mapping to the power set, $\mathcal{P}(\NN)$ which is uncountable, so the set of all functions from $\braces{0, 1}$ to $\NN$ is also uncountable. 
    \end{solution}

    \part Does $\mathcal{P}(\NN)$ contain an uncountable antichain?
    \begin{solution}
        Yes, $\mathcal{P}(\NN)$ does contain an uncountable antichain. For each $x \in (0, 1)$ we represent it as binary: $x = 0.b_{1}b_{2}b_{3}\ldots$ for $b_n \in \braces{0, 1}$. Now take the binary representation of $x$ and map it to subset of $\NN \times \NN$ by the map $(n, b_n)$ for each $b_n$ in the binary representation of $x$ (eg. for $x = 0.101\ldots$ we have $\braces{(1, 1), (2, 0), (3, 1), \ldots}$). Then, for any $x \neq y$ there will be some index, say $n=k$, at which $b_{x_{k}} \neq b_{y_{k}}$. This results in no set being a subset of another. Since there are uncountably many real numbers between $0$ and $1$, there are uncountable many sequences and this results in an uncountable antichain.
    \end{solution}
\end{parts}

\footnotetext{\LaTeX\, code for this document can be found on github \href{https://github.com/jeevanshah07/MATH311/blob/main/homework/homework2/main.tex}{\underline{here}}}
\end{document}